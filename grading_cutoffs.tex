\documentclass{article}

\begin{document}

\subsection{Grading Cutoffs}

As stated \href{https://registrar.ucdavis.edu/records/grades/letter}{here}, letter grades say something about your performance in the class:

\begin{tabular}{|l|l|}
\hline
Grade & Description \\ \hline
A     & ``Excellent''     \\ \hline
B     & ``Good''     \\ \hline
C     & ``Fair''     \\ \hline
D     & ``Barely passing''     \\ \hline
F     & ``Not passing''      \\ \hline
\end{tabular}

\vspace{1em}
The letter grade cutoffs in this class are the standard ones:

\begin{tabular}{|l|l|}
\hline
Grade & Cutoff \\ \hline
A+    & 97     \\ \hline
A     & 93     \\ \hline
A-    & 90     \\ \hline
B+    & 87     \\ \hline
B     & 83     \\ \hline
B-    & 80     \\ \hline
C+    & 77     \\ \hline
C     & 73     \\ \hline
C-    & 70     \\ \hline
D+    & 67     \\ \hline
D     & 63     \\ \hline
D-    & 60     \\ \hline
F     & 0      \\ \hline
\end{tabular}

\vspace{1em}
I do not bump you up to the next grade if you are ``close enough''. For example, if your final percentage at the end of the quarter is 82.99, then you get a B-, not a B.

It is possible that I may make some adjustment such as lowering the grading cutoffs or giving extra credit assignments. Such adjustments will not be made on the basis of the grade distribution. If half the class does a horrible job and earns an F, then so be it. Conversely, if half the class does a fantastic job and earns an A, then that's fine too. I'm not concerned with making sure the average grade lands at a certain spot; I'm concerned with making sure students get the grade they earned. \textbf{You should not count on adjustments being made.}

\end{document}
